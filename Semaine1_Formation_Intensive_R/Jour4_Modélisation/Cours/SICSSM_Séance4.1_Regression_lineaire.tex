\documentclass[ignorenonframetext,]{beamer}
\setbeamertemplate{caption}[numbered]
\setbeamertemplate{caption label separator}{: }
\setbeamercolor{caption name}{fg=normal text.fg}
\beamertemplatenavigationsymbolsempty
\usepackage{lmodern}
\usepackage{amssymb,amsmath}
\usepackage{ifxetex,ifluatex}
\usepackage{fixltx2e} % provides \textsubscript
\ifnum 0\ifxetex 1\fi\ifluatex 1\fi=0 % if pdftex
  \usepackage[T1]{fontenc}
  \usepackage[utf8]{inputenc}
\else % if luatex or xelatex
  \ifxetex
    \usepackage{mathspec}
  \else
    \usepackage{fontspec}
  \fi
  \defaultfontfeatures{Ligatures=TeX,Scale=MatchLowercase}
\fi
\usetheme[]{Antibes}
\usecolortheme{beaver}
\usefonttheme{structurebold}
% use upquote if available, for straight quotes in verbatim environments
\IfFileExists{upquote.sty}{\usepackage{upquote}}{}
% use microtype if available
\IfFileExists{microtype.sty}{%
\usepackage{microtype}
\UseMicrotypeSet[protrusion]{basicmath} % disable protrusion for tt fonts
}{}
\newif\ifbibliography
\hypersetup{
            pdftitle={Séance 4.1: Regression linéaire simple et multiple},
            pdfauthor={Visseho Adjiwanou, PhD.},
            pdfborder={0 0 0},
            breaklinks=true}
\urlstyle{same}  % don't use monospace font for urls
\usepackage{color}
\usepackage{fancyvrb}
\newcommand{\VerbBar}{|}
\newcommand{\VERB}{\Verb[commandchars=\\\{\}]}
\DefineVerbatimEnvironment{Highlighting}{Verbatim}{commandchars=\\\{\}}
% Add ',fontsize=\small' for more characters per line
\usepackage{framed}
\definecolor{shadecolor}{RGB}{248,248,248}
\newenvironment{Shaded}{\begin{snugshade}}{\end{snugshade}}
\newcommand{\KeywordTok}[1]{\textcolor[rgb]{0.13,0.29,0.53}{\textbf{#1}}}
\newcommand{\DataTypeTok}[1]{\textcolor[rgb]{0.13,0.29,0.53}{#1}}
\newcommand{\DecValTok}[1]{\textcolor[rgb]{0.00,0.00,0.81}{#1}}
\newcommand{\BaseNTok}[1]{\textcolor[rgb]{0.00,0.00,0.81}{#1}}
\newcommand{\FloatTok}[1]{\textcolor[rgb]{0.00,0.00,0.81}{#1}}
\newcommand{\ConstantTok}[1]{\textcolor[rgb]{0.00,0.00,0.00}{#1}}
\newcommand{\CharTok}[1]{\textcolor[rgb]{0.31,0.60,0.02}{#1}}
\newcommand{\SpecialCharTok}[1]{\textcolor[rgb]{0.00,0.00,0.00}{#1}}
\newcommand{\StringTok}[1]{\textcolor[rgb]{0.31,0.60,0.02}{#1}}
\newcommand{\VerbatimStringTok}[1]{\textcolor[rgb]{0.31,0.60,0.02}{#1}}
\newcommand{\SpecialStringTok}[1]{\textcolor[rgb]{0.31,0.60,0.02}{#1}}
\newcommand{\ImportTok}[1]{#1}
\newcommand{\CommentTok}[1]{\textcolor[rgb]{0.56,0.35,0.01}{\textit{#1}}}
\newcommand{\DocumentationTok}[1]{\textcolor[rgb]{0.56,0.35,0.01}{\textbf{\textit{#1}}}}
\newcommand{\AnnotationTok}[1]{\textcolor[rgb]{0.56,0.35,0.01}{\textbf{\textit{#1}}}}
\newcommand{\CommentVarTok}[1]{\textcolor[rgb]{0.56,0.35,0.01}{\textbf{\textit{#1}}}}
\newcommand{\OtherTok}[1]{\textcolor[rgb]{0.56,0.35,0.01}{#1}}
\newcommand{\FunctionTok}[1]{\textcolor[rgb]{0.00,0.00,0.00}{#1}}
\newcommand{\VariableTok}[1]{\textcolor[rgb]{0.00,0.00,0.00}{#1}}
\newcommand{\ControlFlowTok}[1]{\textcolor[rgb]{0.13,0.29,0.53}{\textbf{#1}}}
\newcommand{\OperatorTok}[1]{\textcolor[rgb]{0.81,0.36,0.00}{\textbf{#1}}}
\newcommand{\BuiltInTok}[1]{#1}
\newcommand{\ExtensionTok}[1]{#1}
\newcommand{\PreprocessorTok}[1]{\textcolor[rgb]{0.56,0.35,0.01}{\textit{#1}}}
\newcommand{\AttributeTok}[1]{\textcolor[rgb]{0.77,0.63,0.00}{#1}}
\newcommand{\RegionMarkerTok}[1]{#1}
\newcommand{\InformationTok}[1]{\textcolor[rgb]{0.56,0.35,0.01}{\textbf{\textit{#1}}}}
\newcommand{\WarningTok}[1]{\textcolor[rgb]{0.56,0.35,0.01}{\textbf{\textit{#1}}}}
\newcommand{\AlertTok}[1]{\textcolor[rgb]{0.94,0.16,0.16}{#1}}
\newcommand{\ErrorTok}[1]{\textcolor[rgb]{0.64,0.00,0.00}{\textbf{#1}}}
\newcommand{\NormalTok}[1]{#1}
\usepackage{longtable,booktabs}
\usepackage{caption}
% These lines are needed to make table captions work with longtable:
\makeatletter
\def\fnum@table{\tablename~\thetable}
\makeatother

% Prevent slide breaks in the middle of a paragraph:
\widowpenalties 1 10000
\raggedbottom

\AtBeginPart{
  \let\insertpartnumber\relax
  \let\partname\relax
  \frame{\partpage}
}
\AtBeginSection{
  \ifbibliography
  \else
    \let\insertsectionnumber\relax
    \let\sectionname\relax
    \frame{\sectionpage}
  \fi
}
\AtBeginSubsection{
  \let\insertsubsectionnumber\relax
  \let\subsectionname\relax
  \frame{\subsectionpage}
}

\setlength{\parindent}{0pt}
\setlength{\parskip}{6pt plus 2pt minus 1pt}
\setlength{\emergencystretch}{3em}  % prevent overfull lines
\providecommand{\tightlist}{%
  \setlength{\itemsep}{0pt}\setlength{\parskip}{0pt}}
\setcounter{secnumdepth}{0}

\title{Séance 4.1: Regression linéaire simple et multiple}
\author{Visseho Adjiwanou, PhD.}
\institute{SICSS - Montréal}
\date{09 June 2021}

\begin{document}
\frame{\titlepage}

\begin{frame}{Plan de présentation}

\begin{itemize}
\tightlist
\item
  Exemples graphiques
\item
  Exemple 1: accès aux médias et attitude face à la violence
\item
  Exemple 2: Relation entre taille et poids
\item
  Discuter de la meilleure manière d'estimer cette relation
\item
  Présenter le modèle linéaire simple avec ces interprétations
\end{itemize}

\end{frame}

\section{Exemple 1: Ouverture aux médias et attitude face à la
violence}\label{exemple-1-ouverture-aux-muxe9dias-et-attitude-face-uxe0-la-violence}

\begin{frame}[fragile]{Exemple 1:}

Présenter à nouveau la discussion sur la relation entre violence
conjugale et l'accès à l'information mesurée par les variables
\texttt{sec\_school} et \texttt{no\_media}. ouverture aux médias
éducation et radio

\end{frame}

\begin{frame}[fragile]{Exemple}

\begin{longtable}[]{@{}ll@{}}
\toprule
Nom D & escription\tabularnewline
\midrule
\endhead
\texttt{beat\_goesout} & Pourcentage de femmes dans chaque pays qui
pensent qu'un mari\tabularnewline
& a le droit de battre sa femme si elle sort sans le lui
dire.\tabularnewline
\texttt{beat\_burnfood} & Pourcentage de femmes dans chaque pays qui
pensent qu'un mari\tabularnewline
& a le droit de battre sa femme si elle brûle sa
nourriture.\tabularnewline
\texttt{no\_media} & Pourcentage de femmes dans chaque pays qui ont
rarement accès\tabularnewline
& un journal, une radio ou une télévision.\tabularnewline
\texttt{sec\_school} & Pourcentage de femmes dans chaque pays ayant un
niveau\tabularnewline
& d'éducation secondaire ou supérieur.\tabularnewline
\texttt{year} & Année de l'enquête\tabularnewline
\texttt{region} & Région du monde\tabularnewline
\texttt{country} & pays\tabularnewline
\bottomrule
\end{longtable}

\end{frame}

\begin{frame}[fragile]{Dressons la table}

\begin{Shaded}
\begin{Highlighting}[]
\KeywordTok{rm}\NormalTok{(}\DataTypeTok{list =} \KeywordTok{ls}\NormalTok{())}

\KeywordTok{library}\NormalTok{(tidyverse)}
\end{Highlighting}
\end{Shaded}

\begin{verbatim}
## -- Attaching packages --------------------------------------- tidyverse 1.3.0 --
\end{verbatim}

\begin{verbatim}
## v ggplot2 3.3.3     v purrr   0.3.4
## v tibble  3.1.2     v dplyr   1.0.6
## v tidyr   1.1.3     v stringr 1.4.0
## v readr   1.4.0     v forcats 0.4.0
\end{verbatim}

\begin{verbatim}
## Warning: package 'ggplot2' was built under R version 3.6.2
\end{verbatim}

\begin{verbatim}
## Warning: package 'tibble' was built under R version 3.6.2
\end{verbatim}

\begin{verbatim}
## Warning: package 'tidyr' was built under R version 3.6.2
\end{verbatim}

\begin{verbatim}
## Warning: package 'readr' was built under R version 3.6.2
\end{verbatim}

\begin{verbatim}
## Warning: package 'purrr' was built under R version 3.6.2
\end{verbatim}

\begin{verbatim}
## Warning: package 'dplyr' was built under R version 3.6.2
\end{verbatim}

\begin{verbatim}
## -- Conflicts ------------------------------------------ tidyverse_conflicts() --
## x dplyr::filter() masks stats::filter()
## x dplyr::lag()    masks stats::lag()
\end{verbatim}

\begin{Shaded}
\begin{Highlighting}[]
\NormalTok{dhs_ipv <-}\StringTok{ }\KeywordTok{read_csv}\NormalTok{(}\StringTok{"../Données/dhs_ipv.csv"}\NormalTok{)}
\end{Highlighting}
\end{Shaded}

\begin{verbatim}
## Warning: Missing column names filled in: 'X1' [1]
\end{verbatim}

\begin{verbatim}
## 
## -- Column specification --------------------------------------------------------
## cols(
##   X1 = col_double(),
##   beat_burnfood = col_double(),
##   beat_goesout = col_double(),
##   sec_school = col_double(),
##   no_media = col_double(),
##   country = col_character(),
##   year = col_double(),
##   region = col_character()
## )
\end{verbatim}

\end{frame}

\begin{frame}[fragile]{Quelques informations sur les données}

\begin{Shaded}
\begin{Highlighting}[]
\KeywordTok{head}\NormalTok{(dhs_ipv)}
\end{Highlighting}
\end{Shaded}

\begin{verbatim}
## # A tibble: 6 x 8
##      X1 beat_burnfood beat_goesout sec_school no_media country   year region    
##   <dbl>         <dbl>        <dbl>      <dbl>    <dbl> <chr>    <dbl> <chr>     
## 1     1           4.4         18.6       25.2      1.5 Albania   2008 Middle Ea~
## 2     4           4.9         19.9       67.7      8.7 Armenia   2000 Middle Ea~
## 3     5           2.1         10.3       67.6      2.2 Armenia   2005 Middle Ea~
## 4     6           0.3          3.1       46        6.4 Armenia   2010 Middle Ea~
## 5     7          12.1         42.5       74.6      7.4 Azerbai~  2006 Middle Ea~
## 6     8          NA           NA         24       41.9 Banglad~  2004 Asia
\end{verbatim}

\end{frame}

\begin{frame}[fragile]{Quelques informations sur les données}

\begin{Shaded}
\begin{Highlighting}[]
\KeywordTok{glimpse}\NormalTok{(dhs_ipv)}
\end{Highlighting}
\end{Shaded}

\begin{verbatim}
## Rows: 151
## Columns: 8
## $ X1            <dbl> 1, 4, 5, 6, 7, 8, 9, 10, 11, 12, 13, 14, 15, 16, 17, 18,~
## $ beat_burnfood <dbl> 4.4, 4.9, 2.1, 0.3, 12.1, NA, NA, 4.1, 29.2, 19.0, 6.0, ~
## $ beat_goesout  <dbl> 18.6, 19.9, 10.3, 3.1, 42.5, NA, 17.9, 17.3, 44.0, 36.7,~
## $ sec_school    <dbl> 25.2, 67.7, 67.6, 46.0, 74.6, 24.0, 27.3, 27.9, 7.7, 10.~
## $ no_media      <dbl> 1.5, 8.7, 2.2, 6.4, 7.4, 41.9, 45.1, 48.8, 33.2, 38.8, 4~
## $ country       <chr> "Albania", "Armenia", "Armenia", "Armenia", "Azerbaijan"~
## $ year          <dbl> 2008, 2000, 2005, 2010, 2006, 2004, 2007, 2011, 2001, 20~
## $ region        <chr> "Middle East and Central Asia", "Middle East and Central~
\end{verbatim}

\end{frame}

\begin{frame}[fragile]{Quelques informations sur les données}

\begin{Shaded}
\begin{Highlighting}[]
\KeywordTok{summary}\NormalTok{(dhs_ipv)}
\end{Highlighting}
\end{Shaded}

\begin{verbatim}
##        X1         beat_burnfood    beat_goesout     sec_school   
##  Min.   :  1.00   Min.   : 0.10   Min.   : 0.30   Min.   : 3.10  
##  1st Qu.: 40.50   1st Qu.: 4.50   1st Qu.:11.85   1st Qu.:10.18  
##  Median : 79.00   Median :11.85   Median :28.10   Median :22.40  
##  Mean   : 80.53   Mean   :15.04   Mean   :28.60   Mean   :24.40  
##  3rd Qu.:119.50   3rd Qu.:22.25   3rd Qu.:42.08   3rd Qu.:34.90  
##  Max.   :160.00   Max.   :64.50   Max.   :82.70   Max.   :74.60  
##                   NA's   :31      NA's   :27      NA's   :3      
##     no_media       country               year         region         
##  Min.   : 0.80   Length:151         Min.   :1999   Length:151        
##  1st Qu.:11.25   Class :character   1st Qu.:2004   Class :character  
##  Median :29.15   Mode  :character   Median :2007   Mode  :character  
##  Mean   :28.40                      Mean   :2007                     
##  3rd Qu.:43.23                      3rd Qu.:2011                     
##  Max.   :86.40                      Max.   :2014                     
##  NA's   :13
\end{verbatim}

\end{frame}

\begin{frame}[fragile]{Association entre beat\_burnfood et niveau
d'éducation}

\begin{Shaded}
\begin{Highlighting}[]
\KeywordTok{ggplot}\NormalTok{(dhs_ipv) }\OperatorTok{+}
\StringTok{  }\KeywordTok{geom_point}\NormalTok{(}\KeywordTok{aes}\NormalTok{(}\DataTypeTok{x =}\NormalTok{ sec_school, }\DataTypeTok{y =}\NormalTok{ beat_burnfood), }\DataTypeTok{color =} \StringTok{"blue"}\NormalTok{) }\OperatorTok{+}
\StringTok{  }\KeywordTok{geom_text}\NormalTok{(}\KeywordTok{aes}\NormalTok{(}\DataTypeTok{x =}\NormalTok{ sec_school, }\DataTypeTok{y =}\NormalTok{ beat_burnfood, }\DataTypeTok{label =}\NormalTok{ country), }\DataTypeTok{size =} \DecValTok{2}\NormalTok{)}
\end{Highlighting}
\end{Shaded}

\includegraphics[width=0.6\linewidth,height=0.8\textheight]{SICSSM_Séance4.1_Regression_lineaire_files/figure-beamer/unnamed-chunk-5-1}

\end{frame}

\end{document}
